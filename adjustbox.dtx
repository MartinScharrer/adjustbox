% \iffalse meta-comment
%
% Copyright (C) 2011 by Martin Scharrer <martin@scharrer-online.de>
% -----------------------------------------------------------------
%
% This work may be distributed and/or modified under the
% conditions of the LaTeX Project Public License, either version 1.3c
% of this license or (at your option) any later version.
% The latest version of this license is in
%
%   http://www.latex-project.org/lppl.txt
%
% and version 1.3c or later is part of all distributions of LaTeX
% version 2008/05/04 or later.
%
% This work has the LPPL maintenance status `maintained'.
%
% The Current Maintainer of this work is Martin Scharrer.
%
% This work consists of the files adjustbox.dtx, adjustbox.ins
% and the derived file adjustbox.sty.
%
% $Id$
% \fi
%
% \iffalse
%<package>\ProvidesPackage{adjustbox}
%<*driver>
\ProvidesFile{adjustbox.dtx}
%</driver>
  [2011/01/27 v0.2 Adjusting TeX boxes (trim, clip, ...)]
%<*driver>
\documentclass{ydoc}
\GetFileInfo{\jobname.dtx}
\usepackage{adjustbox}[\filedate]
\usepackage{array}
\usepackage{amsmath}
\usepackage{xcolor}
\usepackage{booktabs}

%\EnableCrossrefs
%\CodelineIndex
%\RecordChanges
%\OnlyDescription
\renewcommand{\bottomfraction}{0.5}
\begin{document}
  \DocInput{\jobname.dtx}
  \PrintChanges
  %\newpage\PrintIndex
\end{document}
%</driver>
% \fi
%
% \CheckSum{185}
%
% \CharacterTable
%  {Upper-case    \A\B\C\D\E\F\G\H\I\J\K\L\M\N\O\P\Q\R\S\T\U\V\W\X\Y\Z
%   Lower-case    \a\b\c\d\e\f\g\h\i\j\k\l\m\n\o\p\q\r\s\t\u\v\w\x\y\z
%   Digits        \0\1\2\3\4\5\6\7\8\9
%   Exclamation   \!     Double quote  \"     Hash (number) \#
%   Dollar        \$     Percent       \%     Ampersand     \&
%   Acute accent  \'     Left paren    \(     Right paren   \)
%   Asterisk      \*     Plus          \+     Comma         \,
%   Minus         \-     Point         \.     Solidus       \/
%   Colon         \:     Semicolon     \;     Less than     \<
%   Equals        \=     Greater than  \>     Question mark \?
%   Commercial at \@     Left bracket  \[     Backslash     \\
%   Right bracket \]     Circumflex    \^     Underscore    \_
%   Grave accent  \`     Left brace    \{     Vertical bar  \|
%   Right brace   \}     Tilde         \~}
%
%
% \changes{v0.1}{2011/01/24}{First internal version}
% \changes{v0.2}{2011/01/27}{First released version}
%
% \GetFileInfo{\jobname.dtx}
%
% \DoNotIndex{\newcommand,\newenvironment,\def,\edef,\xdef,\gdef,\let}
%
% \ifpdf
% \hypersetup{%
%   pdfauthor   = {Martin Scharrer <martin@scharrer-online.de>},
%   pdftitle    = {The adjustbox package},
%   pdfsubject  = {Documentation of LaTeX package adjustbox},
%   pdfkeywords = {adjustbox, LaTeX, TeX}
% }%
% \fi
% \clearpage
% \null
% \vspace*{-2em}
% \begin{center}
%   {\huge The \textsf{adjustbox} Package\\[\medskipamount]}
%   {\large Martin Scharrer \\[\medskipamount]\normalsize
%   \url{martin@scharrer-online.de}\\[.8ex]
%   \url{http://www.ctan.org/pkg/adjustbox/}\\[\bigskipamount]}
%   {\large Version \fileversion\ -- \filedate}\\
% \end{center}
% \vspace{1.2em}%
% \def\optstar{\textcolor{optional}{*}}
%
% \makeatletter
% \def\LATeX{(L\kern -.36em{\sbox \z@ T\vbox to\ht \z@ {\hbox {\check@mathfonts
%  \fontsize \sf@size \z@ \math@fontsfalse \selectfont A}\vss }}\kern -.15em)\TeX}
% \makeatother
%
% \begin{abstract}
%  This package provides macros missing in \pkg{graphics/x} to trim, clip and generally adjust boxed \LaTeX{} material.
%  The macros allow for verbatim content. Equivalent environments are also provided. The trim and clip operation
%  are implemented using the \pkg{pgf} package, which supports both DVI/PS and PDF output.
%  \\
%  \textcolor{red}{This package is new and the implementation might change in upcoming releases!}
% \end{abstract}
%
% \section{Introduction}
% The standard \LaTeX{} package \pkg{graphicx} (the extended version of \pkg{graphics}) provides the macro \Macro\includegraphics[<options>]{<file name>} which can
% be used to include graphic files. Several options can be used to scale, resize, rotate, trim and/or clip the graphic.
% The macros \Macro\scalebox, \Macro\resizebox and \Macro\rotatebox are also provided to apply the corresponding 
% operation on \LATeX{} material, which is subsequently placed inside a \Macro\hbox.
% However no macros are provided to trim or clip \LATeX{} material, most likely because this operations
% are not done by \TeX{} but by the output format, i.e. using PostScript (PS) or PDF operations.
% 
% This package provides the missing macros \Macro\clipbox and \Macro\trimbox
% as well as the general \Macro\adjustbox macro. The clipping and trimming operations are implemented using
% a \env{pgfpicture} environment from the \pkg{pgf} package which supports both PS and PDF output.
%
%
% \section{Usage}
% This section describes the usage of the provided macros, which are outlined in section~\ref{sec:boxmacros}.
% Possible advanced values for the macro arguments are mentioned in section~\ref{sec:argval}. The existing verbatim support
% is explained in section~\ref{sec:verbatim}. Finally section~\ref{sec:alternatives} compares the existing macros with the 
% corresponding options of \Macro\adjustbox.
% It is recommended to also read the \emph{Graphics Guide} (|grfguide|, i.e.~the manual of the |graphics|/|x| packages),
% to understand the existing options for \Macro\includegraphics. See the example section for examples of this macros.
%
% \subsection{Box Modification Macros}\label{sec:boxmacros}
%
% \subsubsection*{Trim Box Content}
%
% \vskip-\lastskip
% \DescribeMacro\trimbox*{<llx>~<lly>~<urx>~<ury>}{<text>}
% The macro \Macro\trimbox trims the given amount from the lower left (ll) and the upper right (ur) corner of
% the box. This means that the amount \meta{llx} is trimmed from the left side, \meta{lly} from the bottom and
% \meta{urx} and \meta{ury} from the right and top of the box, respectively.
% Trimming means that the official size of the box is reduced, but no material
% is actual removed. The material in the trimmed areas simply swaps over the official border.
%
% If the starred version is used the four coordinates are taken as the |viewport| instead, i.e. the box
% is trimmed to the rectangle described by the coordinates.
%
%
%
% \DescribeEnv[<text>]{trimbox}*{<llx>~<lly>~<urx>~<ury>}
% \vspace{-\baselineskip}
% \DescribeEnv[<text>]{trimbox*}{<llx>~<lly>~<urx>~<ury>}
% The \env{trimbox} and \env{trimbox*} environments do the same as the corresponding macros.
% Special care is taken so that the macros and the environments can have the same name.
% Because of this the star can be either part of the name or an optional argument.
% Also the plain\TeX{} syntax for environments (|\trimbox ... \endtrimbox|) can not be used
% because it will trigger \Macro\trimbox in macro mode. 
%
%
% \subsubsection*{Clip Box Content}
% \vskip-\lastskip
% \DescribeMacro\clipbox*{<llx>~<lly>~<urx>~<ury>}{<text>}
% The \Macro\clipbox macro works like the \Macro\trimbox and trims the given amounts from the \meta{text}.
% However, in addition the trimmed material is also clipped, i.e. it is not shown in the final document.
% Note that the material will still be part of the output file but is simply not shown.
% It might be exported using special tools, so using \Macro\clipbox\relax (or \Macro\includegraphics[clip,trim=...])
% to censor classified information would be a bad idea.
% The starred version will again use the given coordinates as |viewport|.
%
% \DescribeEnv[<text>]{clipbox}*{<llx>~<lly>~<urx>~<ury>}
% \vspace{-\baselineskip}
% \DescribeEnv[<text>]{clipbox*}{<llx>~<lly>~<urx>~<ury>}
% The environment versions of \Macro\clipbox and \Macro\clipbox*.
%
%
% \subsubsection*{Adjust Box Content}
% \vskip-\lastskip
% \DescribeMacro\adjustbox{<includegraphics options>}{<text>}
% The \Macro\adjustbox macro is the general form of all box modifying macros mentioned in the introduction.
% It can be thought as an \Macro\includegraphics for \LATeX{} material.
% It supports the same set of \meta{options}, however they are provided as a mandatory not as an optional argument.
% An \Macro\adjustbox without options would not make sense and can be replaced by a simple \Macro\mbox.
% There is no starred version of this macro. See also Table~\ref{tab:alternatives} for a comparison of \Macro\adjustbox
% with the other macros.
%
% \DescribeEnv[<text>]{adjustbox}{<includegraphics options>}
% The environment version of \Macro\adjustbox.
%
%
% \subsection{Argument Values}\label{sec:argval}
%
% The\marginpar{Parsing} argument values are parsed by versatile \Macro\pgfmathparse of the already used |pgf| package. See the |pgfmanual| for detailed information.
% This allows very complex arithmetic expressions as any of the trim/clip coordinates or other numeric options.
% Note\marginpar{Space=Separator} that the four values for \Macro\trimbox and \Macro\clipbox as well as for the |trim| and |viewport| option of \Macro\adjustbox
% are separated by spaces. If the expression of any of this values holds a space or ends with a macro (eats trailing spaces!) it must be wrapped into braces `|{ }|'.
%
% \DescribeMacros
%    \hbox{\Macro\width~~~\Macro\height~~~\Macro\depth~~~\Macro\totalheight}%
% \endDescribeMacros
% This \LaTeX{} lengths hold the original dimension of \meta{text} and can be used as part of the arguments to \Macro\adjustbox, \Macro\trimbox and \Macro\clipbox.
% The totalheight is the height plus depth.
%
% If\marginpar{Default unit} no unit is provided for of the bounding box coordinates (llx, lly, urx, ury) then PostScript points
% (\emph{big points}, bp, $72\,\text{bp}=1\,\text{inch}$) are used, as it is the default behaviour
% of the |trim| and |viewport| ptions of \pkg{graphicx}'s \Macro\includegraphics. Note that \pkg{graphicx} converts all values, independent if a unit is provided or not,
% internally to bp, because graphics where traditionally stored in Encapsulated PostScript (EPS) files. The more modern PDF files also use bp instead of pt.
% Because the |adjustbox| package macros target \LATeX{} material and users will mostly use pt values this internal conversion to bp got disabled for them to 
% avoid unnecessary rounding errors.
%
%
% \subsubsection*{Examples for Argument Values}
% \hbox to \linewidth{\hss\Macro\trimbox{'{.5\width} 10 {log10(10)/sin(45) + 1} 10pt'}{<content>}\hss}
% will trim half the original amount from the left, 10\,bp from the bottom and 
% 2.42328\,bp from the right (bp, because no unit was used in the formula; change e.g. `|+ 1|' to `|+ 1pt|' to get 2.42328\,pt),
% as well as 10\,pt from the top.
%
% \subsection{Verbatim Support}\label{sec:verbatim}
% The macros read the \meta{text} as \TeX{} \Macro\hbox and not as an macro argument in order to support verbatim content.
% This means that the braces around the content can also be written as \Macro\bgroup and \Macro\egroup:\\[\smallskipamount]
% \hspace*{2\parindent}\Macro\trimbox{1 2 3 4}\AlsoMacro\bgroup~<content>\AlsoMacro\egroup\\[\smallskipamount]
% Special care is taken to allow the \meta{text} to be a single macro (except \Macro\bgroup) without any braces:\\[\smallskipamount]
% \hspace*{2\parindent}\Macro\clipbox{1 2 3 4}\AlsoMacro\somemacro\\[\smallskipamount]
% This is to support the questionable habit of some \LaTeX{} users to drop the braces for single token arguments.
% All environments support verbatim content.
%
% \subsection{Alternatives for existing Macros}\label{sec:alternatives}
% The flexible \Macro\adjustbox can also be used as an alternative to existing macros from the \pkg{graphics} package as shown by Table~\ref{tab:alternatives}.
% Because it is longer then the originals this is only of benefit if combinations are to be replaced or verbatim text must be supported.
%
% \begin{table}[b]
% \belowcaptionskip\abovecaptionskip
% \colorlet{optional}{black}
% \caption{Alternatives for existing Macros}\label{tab:alternatives}
% \hbox to \linewidth{\hss
% \begin{tabular}{ll}
%   \toprule
%   Original Macro (w/o text argument)  &  Alternative  (w/o text argument) \\
%   \midrule
%   \Macro\rotatebox{<angle>}    &   \Macro\adjustbox{'angle='<angle>}  \\
%   \Macro\scalebox{<factor>}    &   \Macro\adjustbox{'scale='<factor>}  \\
%   \Macro\scalebox{<x-factor>}[<y-factor>]    &   \Macro\adjustbox{'width='<x-factor>\AlsoMacro\width',height='<y-factor>\AlsoMacro\height}  \\
%   \Macro\reflectbox            &   \Macro\adjustbox{'width=-'\AlsoMacro\width',height='\AlsoMacro\height}  \\
%   \Macro\resizebox{<width>}{<height>}    &   \Macro\adjustbox{'width='<width>',height='<height>}  \\
%   \Macro\resizebox*{<width>}{<totalheight>}    &   \Macro\adjustbox{'width='<width>',totalheight='<totalheight>}  \\
%   \Macro\trimbox{<llx>~<lly>~<urx>~<ury>}    &   \Macro\adjustbox{'trim='<llx>~<lly>~<urx>~<ury>}  \\
%   \Macro\trimbox*{<llx>~<lly>~<urx>~<ury>}    &   \Macro\adjustbox{'viewport='<llx>~<lly>~<urx>~<ury>}  \\
%   \Macro\clipbox{<llx>~<lly>~<urx>~<ury>}    &   \Macro\adjustbox{'trim='<llx>~<lly>~<urx>~<ury>,clip}  \\
%   \Macro\clipbox*{<llx>~<lly>~<urx>~<ury>}    &   \Macro\adjustbox{'viewport='<llx>~<lly>~<urx>~<ury>,clip}  \\
%   \bottomrule
% \end{tabular}%
% \hss}%
% \end{table}
%
% \clearpage
% \section{Examples}
%
% \def\examplecontent{\begin{tabular}{@{}|c|c|@{}}
%       \hline
%       A & B \\
%       \hline
%       C & D \\
%       \hline
%   \end{tabular}}
% The following examples show the application of the package macros on an example text.
% The result is placed in a tight, colored frame box to show the resulting dimensions.
% \begingroup
% \fboxsep=0pt%
% \def\Fbox{\fcolorbox{red}{white}}%
% \def\X{\vspace*{20pt}}%
% \par\bigskip\noindent
% \begin{tabular}{@{}lc}
%   \X\Macro\example      & \Fbox{\examplecontent} \\
%   \X\Macro\trimbox{10 5 10 5}{\AlsoMacro\example} & \Fbox{\trimbox{10 5 10 5}{\examplecontent}} \\
%   \X\Macro\clipbox{10 5 10 5}{\AlsoMacro\example} & \Fbox{\clipbox{10 5 10 5}{\examplecontent}} \\
%   \X\Macro\trimbox*{15 5 25 30}{\AlsoMacro\example} & \Fbox{\trimbox*{15 5 25 30}{\examplecontent}} \\
%   \X\Macro\clipbox*{15 5 25 30}{\AlsoMacro\example} & \Fbox{\clipbox*{15 5 25 30}{\examplecontent}} \\
%   \X\Macro\adjustbox{trim=10 5 10 5,angle=45}{\AlsoMacro\example} & \Fbox{\adjustbox{trim=10 5 10 5,angle=45}{\examplecontent}} \\
%   \X\Macro\adjustbox{scale=1.5}{\AlsoMacro\example} & \Fbox{\adjustbox{scale=1.5}{\examplecontent}} \\
%   \X\Macro\adjustbox{width=180pt,height=20pt}{\AlsoMacro\example} & \Fbox{\adjustbox{width=40pt,height=10pt}{\examplecontent}} \\
%   \X\Macro\adjustbox{width=180pt,height=20pt,keepaspectratio}{\AlsoMacro\example} & \Fbox{\adjustbox{width=40pt,height=10pt,keepaspectratio}{\examplecontent}} \\
% \end{tabular}
%
% \subsubsection*{Environment example:}
% \noindent
% \Macro\begin{adjustbox}{angle=2}:\\
% \begin{adjustbox}{angle=2}
%    | verbatim inside \begin{adjustbox}{angle=2} ... \end{adjustbox} |
% \end{adjustbox}
% \endgroup
%
% \StopEventually{}
% \clearpage
% \iffalse
%<*package>
% \fi
% \section{Implementation}
%
%    \begin{macrocode}
\RequirePackage{graphicx}[1999/02/16]
\RequirePackage{pgf}
%    \end{macrocode}
%
% \begin{environment}{clipbox}
% \begin{macro}{\clipbox}
%    \begin{macrocode}
\newcommand\clipbox{%
    \begingroup
    \def\adjustbox@name{clipbox}%
    \@ifstar
        {\adjustbox@{clip,viewport=}}%
        {\adjustbox@{clip,trim=}}%
}
%    \end{macrocode}
% \end{macro}
%    \begin{macrocode}
\def\endclipbox{%
    \endadjustbox
}
%    \end{macrocode}
% \end{environment}
%
% \begin{environment}{clipbox*}
%    \begin{macrocode}
\newenvironment{clipbox*}
    {\begin{clipbox}*}
    {\end{clipbox}}
%    \end{macrocode}
% \end{environment}
%
% \begin{environment}{trimbox}
% \begin{macro}{\trimbox}
%    \begin{macrocode}
\newcommand\trimbox{%
    \begingroup
    \def\adjustbox@name{trimbox}%
    \@ifstar
        {\adjustbox@{viewport=}}%
        {\adjustbox@{trim=}}%
}
%    \end{macrocode}
% \end{macro}
%    \begin{macrocode}
\def\endtrimbox{%
    \endadjustbox
}
%    \end{macrocode}
% \end{environment}
%
% \begin{environment}{trimbox*}
%    \begin{macrocode}
\newenvironment{trimbox*}
    {\begin{trimbox}*}
    {\end{trimbox}}
%    \end{macrocode}
% \end{environment}
%
% \begin{environment}{adjustbox}
% \begin{macro}{\adjustbox}
%    \begin{macrocode}
\newcommand\adjustbox{%
    \begingroup
    \tracinggroups=1%
    \def\adjustbox@name{adjustbox}%
    \adjustbox@{}%
}
%    \end{macrocode}
% \end{macro}
%    \begin{macrocode}
\def\endadjustbox{%
    \unskip
    \egroup
    \color@endgroup
    \egroup
    \adjustbox@@
}
%    \end{macrocode}
% \end{environment}
%
%
% \begin{macro}{\adjustbox@}
%    \begin{macrocode}
\def\adjustbox@#1#2{%
    \def\adjustbox@setkeys{\setkeys{Gin}{#1#2}}%
    \ifx\@currenvir\adjustbox@name
        \expandafter\def\expandafter\@currenvir\expandafter{\@currenvir\empty}%
        \def\next{%
            \setbox\@tempboxa\hbox\bgroup
                \color@setgroup\bgroup
                \ignorespaces
        }%
    \else
        \def\next{%
            \setbox\@tempboxa\hbox\bgroup%0
                \color@setgroup\bgroup%
                \aftergroup\color@endgroup
                \aftergroup\egroup%
                \aftergroup\adjustbox@@
                \@ifnextchar\bgroup
                    {\let\@let@token=}%
                    {\adjust@box}%
        }%
    \fi
    \next
}
%    \end{macrocode}
% \end{macro}
%
% \begin{macro}{\adjust@box}
%    \begin{macrocode}
\def\adjust@box#1{%
    #1\egroup
}
%    \end{macrocode}
% \end{macro}
%
% \begin{macro}{\adjustbox@@}
%    \begin{macrocode}
\def\adjustbox@@{%
%    \end{macrocode}
% Set the box dimension macros.
%    \begin{macrocode}
    \def\width{\wd\@tempboxa}%
    \def\height{\ht\@tempboxa}%
    \def\depth{\dp\@tempboxa}%
    \@tempdimc=\ht\@tempboxa
    \advance\@tempdimc by \dp\@tempboxa\relax
    \def\totalheight{\@tempdimc}%
%    \end{macrocode}
% Locally redefine \Macro\Gin@defaultbp to use \Macro\pgfmathsetmacro
% with |bp| as the default unit.
% This should yield the same results (apart of smaller rounding errors) if values
% are given without unit but avoids the internal conversion to |bp| of values with
% units.
%    \begin{macrocode}
    \def\pgfmathresultunitscale{1bp}%
    \let\pgfmathpostparse\pgfmathscaleresult
    \let\Gin@defaultbp\pgfmathsetmacro
    \let\setlength\pgfmathsetlength
%    \end{macrocode}
% The rest of the code was adapted from the \Macro\Gin@ii macro from the |graphicx| package.
% The \emph{temp switch a} is set to |true| to indicate to |graphicx| that the
% scaling should be done internal, so this package doesn't have to do it.
% The content including macro \Macro\adjustbox@@@ is but into place,
% the saved options are activated and the final size is set.
% The typesetting of the content is finally done by executing the token register.
%    \begin{macrocode}
    \@tempswatrue
    \toks@{{\adjustbox@@@}}%
    \adjustbox@setkeys
    \Gin@esetsize
    \the\toks@
    \endgroup
}
%    \end{macrocode}
% \end{macro}
%
% \begin{macro}{\adjustbox@@@}
%    \begin{macrocode}
\def\adjustbox@@@{%
    \def\Gin@llx{0}%
    \Gin@defaultbp\Gin@lly{+-\dp\@tempboxa}%
    \Gin@defaultbp\Gin@urx{+\wd\@tempboxa}%
    \Gin@defaultbp\Gin@ury{+\ht\@tempboxa}%
    \Gin@viewport@code
    \begin{pgfpicture}%
        \pgfpathmoveto{\pgfqpoint{\Gin@llx pt}{\Gin@lly pt}}%
        \pgfpathlineto{\pgfqpoint{\Gin@urx pt}{\Gin@lly pt}}%
        \pgfpathlineto{\pgfqpoint{\Gin@urx pt}{\Gin@ury pt}}%
        \pgfpathlineto{\pgfqpoint{\Gin@llx pt}{\Gin@ury pt}}%
        \pgfpathclose
        \expandafter\pgfusepath\ifGin@clip{clip}\else{use as bounding box}\fi\relax
        \pgfset{inner sep=\z@,outer sep=\z@}%
        \pgfnode{rectangle}{base west}{\box\@tempboxa}{}{}%
    \end{pgfpicture}%
}
%    \end{macrocode}
% \end{macro}
%
% \Finale
% \iffalse
%</package>
% \fi
