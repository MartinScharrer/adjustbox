% \iffalse meta-comment
%
% Copyright (C) 2011 by Martin Scharrer <martin@scharrer-online.de>
% -----------------------------------------------------------------
%
% This work may be distributed and/or modified under the
% conditions of the LaTeX Project Public License, either version 1.3c
% of this license or (at your option) any later version.
% The latest version of this license is in
%
%   http://www.latex-project.org/lppl.txt
%
% and version 1.3c or later is part of all distributions of LaTeX
% version 2008/05/04 or later.
%
% This work has the LPPL maintenance status `maintained'.
%
% The Current Maintainer of this work is Martin Scharrer.
%
% This work consists of the files adjustbox.dtx, adjustbox.ins
% and the derived file adjustbox.sty.
%
% $Id$
% \fi
%
% \iffalse
%<package>\ProvidesPackage{adjustbox}
%<*driver>
\ProvidesFile{adjustbox.dtx}
%</driver>
  [2011/01/24 v0.1 Adjusting TeX boxes]
%<*driver>
\documentclass{ydoc}
\GetFileInfo{\jobname.dtx}
\usepackage{adjustbox}[\filedate]
\usepackage{array}
\usepackage{amsmath}
\usepackage{xcolor}

%\EnableCrossrefs
%\CodelineIndex
%\RecordChanges
%\OnlyDescription
\begin{document}
  \DocInput{\jobname.dtx}
  \PrintChanges
  %\newpage\PrintIndex
\end{document}
%</driver>
% \fi
%
% \CheckSum{0}
%
% \CharacterTable
%  {Upper-case    \A\B\C\D\E\F\G\H\I\J\K\L\M\N\O\P\Q\R\S\T\U\V\W\X\Y\Z
%   Lower-case    \a\b\c\d\e\f\g\h\i\j\k\l\m\n\o\p\q\r\s\t\u\v\w\x\y\z
%   Digits        \0\1\2\3\4\5\6\7\8\9
%   Exclamation   \!     Double quote  \"     Hash (number) \#
%   Dollar        \$     Percent       \%     Ampersand     \&
%   Acute accent  \'     Left paren    \(     Right paren   \)
%   Asterisk      \*     Plus          \+     Comma         \,
%   Minus         \-     Point         \.     Solidus       \/
%   Colon         \:     Semicolon     \;     Less than     \<
%   Equals        \=     Greater than  \>     Question mark \?
%   Commercial at \@     Left bracket  \[     Backslash     \\
%   Right bracket \]     Circumflex    \^     Underscore    \_
%   Grave accent  \`     Left brace    \{     Vertical bar  \|
%   Right brace   \}     Tilde         \~}
%
%
% \changes{v0.1}{2011/01/24}{First released version}
%
% \GetFileInfo{\jobname.dtx}
%
% \DoNotIndex{\newcommand,\newenvironment,\def,\edef,\xdef,\gdef,\let}
%
% \ifpdf
% \hypersetup{%
%   pdfauthor   = {Martin Scharrer <martin@scharrer-online.de>},
%   pdftitle    = {The adjustbox package},
%   pdfsubject  = {Documentation of LaTeX package adjustbox},
%   pdfkeywords = {adjustbox, LaTeX, TeX}
% }%
% \fi
% \clearpage
% \null
% \vspace*{-2em}
% \begin{center}
%   {\huge The \textsf{adjustbox} Package\\[\medskipamount]}
%   {\large Martin Scharrer \\[\medskipamount]\normalsize
%   \url{martin@scharrer-online.de}\\[.8ex]
%   \url{http://www.ctan.org/pkg/adjustbox/}\\[\bigskipamount]}
%   {\large Version \fileversion\ -- \filedate}\\
% \end{center}
% \vspace{1.2em}%
% \def\optstar{\textcolor{optional}{*}}
%
% \makeatletter
% \def\LATeX{(L\kern -.36em{\sbox \z@ T\vbox to\ht \z@ {\hbox {\check@mathfonts
%  \fontsize \sf@size \z@ \math@fontsfalse \selectfont A}\vss }}\kern -.15em)\TeX}
% \makeatother
%
% \section{Introduction}
% The standard \LaTeX{} package \pkg{graphicx} (the extended version of \pkg{graphics}) provides the macro \Macro\includegraphics[<options>]{<file name>} which can
% be used to include graphic files. Several options can be used to scale, resize, rotate, trim and/or clip the graphic.
% The macros \Macro\scalebox, \Macro\resizebox and \Macro\rotatebox are also provided to apply the corresponding 
% operation on \LATeX{} material, which is subsequently placed inside a \Macro\hbox.
% However no macros are provided to trim or clip \LATeX{} material, most likely because this operations
% are not done by \TeX{} but by the output format, i.e. using PostScript (PS) or PDF operations.
% 
% This package provides the missing macros \Macro\clipbox and \Macro\trimbox
% as well as the general \Macro\adjustbox macro. The clipping and trimming operations are implemented using
% a \env{pgfpicture} environment from the \pkg{pgf} package which supports both PS and PDF output.
%
% \section{Usage}
% This section describes the usage of the provided macros.
% It is recommended to also read the \emph{Graphics Guide} (|grfguide|, i.e.~the manual of the |graphics|/|graphicx| packages),
% to understand the existing options for \Macro\includegraphics. See the example section for examples of this macros.
%
% \subsection*{Trim Box Content}
% \DescribeMacro\trimbox*{<llx>~<lly>~<urx>~<ury>}{<box content>}
% The macro \Macro\trimbox trims the given amount from the lower left (ll) and the upper right (ur) corner of
% the box. This means that the amount \meta{llx} is trimmed from the left side, \meta{lly} from the bottom and
% \meta{urx} and \meta{ury} from the right and top of the box, respectively.
% Trimming means that the official size of the box is reduced, but no material
% is actual removed. The material in the trimmed areas simply swaps over the official border.
%
% If no unit is provided for any trim amount then PostScript points (\emph{big points}, bp, $72\,\text{bp}=1\,\text{inch}$) are used, as it is the default behaviour
% of the |trim| option of \pkg{graphicx}'s \Macro\includegraphics. Note that \pkg{graphicx} converts all values, independent if a unit is provided or not,
% internally to bp, because graphics where traditionally stored in Encapsulated PostScript (EPS) files. The more modern PDF files also use bp instead of pt.
% Because the |adjustbox| package macros target \LATeX{} material and users will mostly use pt values this internal conversion to bp got disabled for them to 
% avoid unnecessary rounding errors. Instead the values are parsed using the \Macro\pgfmathparse macro from \pkg{pgf} which permits the usage of all
% arithmetic operations supported by it. See the |pgfmanual| for more details on this.
%
% In addition the dimension macros \Macro\width, \Macro\height, \Macro\depth and \Macro\totalheight\relax (=height+depth) can be used as part of the trim amounts.
% The stand for the original dimensions of the \meta{box content}.
% Because a space acts as separator of the amounts any expression which contains a space or ends with an macro must be wrapped into braces `|{ }|'.
% For example
% \\\hbox to \linewidth{\hss\Macro\trimbox{'{.5\width} 10 {log10(10)/sin(45) + 1} 10pt'}{<content>}\hss}
% will trim half the original amount from the left and 
% 2.42328\,bp from the right (bp, because no unit was used in the formula; change e.g. |+ 1| to |+ 1pt| to get 2.42328\,pt),
% as well as 10\,bp from the bottom and 10\,pt from the top.
%
% If the starred version is used the four coordinates are taken as the |viewport| instead, i.e. the box
% is trimmed to the rectangle described by the coordinates.
% 
% Note that like other \LaTeX{} box macros the \meta{box content} is read as an macro argument and therefore
% can not include verbatim material.
%
% \DescribeEnv[<box content, incl.~verbatim material>]{trimbox}{<llx>~<lly>~<urx>~<ury>}
% \vspace{-\baselineskip}
% \DescribeEnv[<box content, incl.~verbatim material>]{trimbox*}{<llx>~<lly>~<urx>~<ury>}
% The \env{trimbox} and \env{trimbox*} environments do the same as the corresponding macros
% but do not read the \meta{box content} as an macro argument and therefore correctly handle 
% catcode changes required by verbatim and other special material.
%
% Note that \LaTeX{} environments, e.g.~\env{example}, are implemented using the two macros
% |\example| and |\endexample| which are used by \Macro\begin and \Macro\end.
% Therefore there can normally not be a macro and an environment with the same name.
% This package defines \Macro\trimbox in a special way to first recognise if it was used 
% by itself as a macro or by \Macro\begin as an environment.
% As an result the \MacroArgs* can also be written as an argument for the \env{trimbox} environment.
% The plain\TeX{} syntax for environments (|\trimbox ... \endtrimbox|) can not be used
% because it will trigger \Macro\trimbox in macro mode. 
%
% \subsection*{Clip Box Content}
% \DescribeMacro\clipbox*{<llx>~<lly>~<urx>~<ury>}{<box content>}
% The \Macro\clipbox macro works like the \Macro\trimbox and trims the given amounts from the box content.
% However, in addition the trimmed material is also clipped, i.e. it is not shown in the final document.
% Note that the material will still be part of the output file but is simply not shown.
% It might be exported using special tools, so using \Macro\clipbox\relax (or \Macro\includegraphics[clip,trim=...])
% to censor classified information would be a bad idea.
% The starred version will again use the given coordinates as |viewport|.
% All other information mentioned for \Macro\trimbox also applies for \Macro\clipbox.
%
% \DescribeEnv[<box content, incl.~verbatim material>]{clipbox}{<llx>~<lly>~<urx>~<ury>}
% \vspace{-\baselineskip}
% \DescribeEnv[<box content, incl.~verbatim material>]{clipbox*}{<llx>~<lly>~<urx>~<ury>}
% The environment versions of \Macro\clipbox and \Macro\clipbox*.
%
%
% \subsection*{Adjust Box Content}
% \DescribeMacro\adjustbox{<includegraphics options>}{<box content>}
% The \Macro\adjustbox macro is the general form of all box modifying macros mentioned in the introduction.
% It can be thought as an \Macro\includegraphics for \LATeX{} material.
% It supports the same set of \meta{options}, however they are provided as a mandatory not as an optional argument.
% An \Macro\adjustbox without options would not make sense and can be replaced by a simple \Macro\mbox.
% There is no
% 
% The macros \Macro\trimbox{\ldots} and \Macro\clipbox{\ldots} can also be written as
% \Macro\adjustbox{trim=\ldots} and \Macro\adjustbox{clip,trim=\ldots}, respectively.
% For the starred versions the |trim| must be replaced by |viewport|.
% Similar the macros \Macro\resizebox{<width>}{<height>}, \Macro\rotatebox{<angle>} and \Macro\scalebox{<factor>} can now be written
% in the form \Macro\adjustbox{'width='<width>',height='<height>}, \Macro\adjustbox{'angle='<angle>} and
% \Macro\adjustbox{'scale='<factor>}, respectively.
% The options can be combined and are applied in the order they are given, i.e.~from left to right.
%
% \DescribeEnv[<box content, incl.~verbatim material>]{adjustbox}{<includegraphics options>}
% The environment version of \Macro\adjustbox.
%
% \clearpage
% \section{Examples}
%
% \def\examplecontent{\begin{tabular}{@{}|c|c|@{}}
%       \hline
%       A & B \\
%       \hline
%       C & D \\
%       \hline
%   \end{tabular}}
% The following examples show the application of the package macros on an example box content.
% The result is placed in a tight, colored frame box to show the resulting dimensions.
% \begingroup
% \fboxsep=0pt%
% \def\Fbox{\fcolorbox{red}{white}}%
% \def\X{\vspace*{20pt}}%
% \par\bigskip\noindent
% \begin{tabular}{@{}lc}
%   \X\Macro\example      & \Fbox{\examplecontent} \\
%   \X\Macro\trimbox{10 5 10 5}{\AlsoMacro\example} & \Fbox{\trimbox{10 5 10 5}{\examplecontent}} \\
%   \X\Macro\clipbox{10 5 10 5}{\AlsoMacro\example} & \Fbox{\clipbox{10 5 10 5}{\examplecontent}} \\
%   \X\Macro\trimbox*{15 5 25 30}{\AlsoMacro\example} & \Fbox{\trimbox*{15 5 25 30}{\examplecontent}} \\
%   \X\Macro\clipbox*{15 5 25 30}{\AlsoMacro\example} & \Fbox{\clipbox*{15 5 25 30}{\examplecontent}} \\
%   \X\Macro\adjustbox{trim=10 5 10 5,angle=45}{\AlsoMacro\example} & \Fbox{\adjustbox{trim=10 5 10 5,angle=45}{\examplecontent}} \\
%   \X\Macro\adjustbox{scale=1.5}{\AlsoMacro\example} & \Fbox{\adjustbox{scale=1.5}{\examplecontent}} \\
%   \X\Macro\adjustbox{width=180pt,height=20pt}{\AlsoMacro\example} & \Fbox{\adjustbox{width=40pt,height=10pt}{\examplecontent}} \\
%   \X\Macro\adjustbox{width=180pt,height=20pt,keepaspectratio}{\AlsoMacro\example} & \Fbox{\adjustbox{width=40pt,height=10pt,keepaspectratio}{\examplecontent}} \\
% \end{tabular}
%
% \subsubsection*{Environment example:}
% \noindent
% \Macro\begin{adjustbox}{angle=2}:\\
% \begin{adjustbox}{angle=2}
%    | verbatim inside \begin{adjustbox}{angle=2} ... \end{adjustbox} |
% \end{adjustbox}
% \endgroup
%
% \StopEventually{}
% \clearpage
% \iffalse
%<*package>
% \fi
% \section{Implementation}
%
%    \begin{macrocode}
\RequirePackage{graphicx}[1999/02/16]
\RequirePackage{pgf}
%    \end{macrocode}
%
% \begin{environment}{clipbox}
% \begin{macro}{\clipbox}
%    \begin{macrocode}
\newcommand\clipbox{%
    \begingroup
    \def\adjustbox@name{clipbox}%
    \@ifstar
        {\adjustbox@{clip,viewport=}}%
        {\adjustbox@{clip,trim=}}%
}
%    \end{macrocode}
% \end{macro}
%    \begin{macrocode}
\def\endclipbox{%
    \egroup
    \color@endgroup
    \egroup
    \adjustbox@@
}
%    \end{macrocode}
% \end{environment}
%
% \begin{environment}{clipbox*}
%    \begin{macrocode}
\newenvironment{clipbox*}
    {\begin{clipbox}*}
    {\end{clipbox}}
%    \end{macrocode}
% \end{environment}
%
% \begin{environment}{trimbox}
% \begin{macro}{\trimbox}
%    \begin{macrocode}
\newcommand\trimbox{%
    \begingroup
    \def\adjustbox@name{trimbox}%
    \@ifstar
        {\adjustbox@{viewport=}}%
        {\adjustbox@{trim=}}%
}
%    \end{macrocode}
% \end{macro}
%    \begin{macrocode}
\let\endtrimbox\endclipbox
%    \end{macrocode}
% \end{environment}
%
% \begin{environment}{trimbox*}
%    \begin{macrocode}
\newenvironment{trimbox*}
    {\begin{trimbox}*}
    {\end{trimbox}}
%    \end{macrocode}
% \end{environment}
%
% \begin{environment}{adjustbox}
% \begin{macro}{\adjustbox}
%    \begin{macrocode}
\newcommand\adjustbox{%
    \begingroup
    \tracinggroups=1%
    \def\adjustbox@name{adjustbox}%
    \adjustbox@{}%
}
%    \end{macrocode}
% \end{macro}
%    \begin{macrocode}
\let\endadjustbox\endclipbox
%    \end{macrocode}
% \end{environment}
%
%
% \begin{macro}{\adjustbox@}
%    \begin{macrocode}
\def\adjustbox@#1#2{%
    \def\adjustbox@setkeys{\setkeys{Gin}{#1#2}}%
    \ifx\@currenvir\adjustbox@name
        \expandafter\def\expandafter\@currenvir\expandafter{\@currenvir\empty}%
        \def\next{%
            \setbox\@tempboxa\hbox\bgroup
                \color@setgroup\bgroup
        }%
    \else
        \def\next{%
            \setbox\@tempboxa\hbox\bgroup%0
                \color@setgroup\bgroup%
                \aftergroup\color@endgroup
                \aftergroup\egroup%
                \aftergroup\adjustbox@@
                \@ifnextchar\bgroup
                    {\let\@let@token=}%
                    {\adjust@box}%
        }%
    \fi
    \next
}
%    \end{macrocode}
% \end{macro}
%
% \begin{macro}{\adjustbox@@@}
%    \begin{macrocode}
\def\adjust@box#1{%
    #1\egroup
}
%    \end{macrocode}
% \end{macro}
%
% \begin{macro}{\adjustbox@@}
%    \begin{macrocode}
\def\adjustbox@@{%
%    \end{macrocode}
% Set the box dimension macros.
%    \begin{macrocode}
    \def\width{\wd\@tempboxa}%
    \def\height{\ht\@tempboxa}%
    \def\depth{\dp\@tempboxa}%
    \@tempdimc=\ht\@tempboxa
    \advance\@tempdimc by \dp\@tempboxa\relax
    \def\totalheight{\@tempdimc}%
%    \end{macrocode}
% Locally redefine \Macro\Gin@defaultbp to use \Macro\pgfmathsetmacro
% with |bp| as the default unit.
% This should yield the same results (apart of smaller rounding errors) if values
% are given without unit but avoids the internal conversion to |bp| of values with
% units.
%    \begin{macrocode}
    \def\pgfmathresultunitscale{1bp}%
    \let\pgfmathpostparse\pgfmathscaleresult
    \let\Gin@defaultbp\pgfmathsetmacro
    \let\setlength\pgfmathsetlength
%    \end{macrocode}
% The rest of the code was adapted from the \Macro\Gin@ii macro from the |graphicx| package.
% The \emph{temp switch a} is set to |true| to indicate to |graphicx| that the
% scaling should be done internal, so this package doesn't have to do it.
% The content including macro \Macro\adjustbox@@@ is but into place,
% the saved options are activated and the final size is set.
% The typesetting of the content is finally done by executing the token register.
%    \begin{macrocode}
    \@tempswatrue
    \toks@{{\adjustbox@@@}}%
    \adjustbox@setkeys
    \Gin@esetsize
    \the\toks@
    \endgroup
}
%    \end{macrocode}
% \end{macro}
%
% \begin{macro}{\adjustbox@@@}
%    \begin{macrocode}
\def\adjustbox@@@{%
    \def\Gin@llx{0}%
    \Gin@defaultbp\Gin@lly{+-\dp\@tempboxa}%
    \Gin@defaultbp\Gin@urx{+\wd\@tempboxa}%
    \Gin@defaultbp\Gin@ury{+\ht\@tempboxa}%
    \Gin@viewport@code
    \begin{pgfpicture}%
        \pgfpathmoveto{\pgfqpoint{\Gin@llx pt}{\Gin@lly pt}}%
        \pgfpathlineto{\pgfqpoint{\Gin@urx pt}{\Gin@lly pt}}%
        \pgfpathlineto{\pgfqpoint{\Gin@urx pt}{\Gin@ury pt}}%
        \pgfpathlineto{\pgfqpoint{\Gin@llx pt}{\Gin@ury pt}}%
        \pgfpathclose
        \expandafter\pgfusepath\ifGin@clip{clip}\else{use as bounding box}\fi\relax
        \pgfset{inner sep=\z@,outer sep=\z@}%
        \pgfnode{rectangle}{base west}{\box\@tempboxa}{}{}%
    \end{pgfpicture}%
}
%    \end{macrocode}
% \end{macro}
%
% \Finale
% \iffalse
%</package>
% \fi
