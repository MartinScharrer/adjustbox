%<*standalone>
\documentclass{standalone}

\usepackage{tikz}
\usetikzlibrary{calc}
\usepackage{adjustbox}
\newsavebox\mybox

\begin{document}
%</standalone>
% Diagram of the TeX box model and its dimensions
% Copyright (C) 2001 by Martin Scharrer <martin.scharrer@web.de>, Nov 12th 2011
% This is free code under the LPPL v1.3 or later version OR the CC BY-SA 3.0

% Save example text in box
\def\text{\sffamily\scalebox{10}{Xy}}
\sbox\mybox{\sffamily\color{black!25}\scalebox{10}{Xy}}
\def\HEIGHT{\ht\mybox}
\def\WIDTH{\wd\mybox}
\def\DEPTH{\dp\mybox}
\def\LLX{.15\WIDTH}
\def\LLY{\DEPTH-.15\HEIGHT}
\def\URX{.15\WIDTH}
\def\URY{.25\HEIGHT}
%
%\usebox\mybox
%\rlap{\trimbox{{\LLX} {\LLY} {\URX} {\URY}}{\usebox\mybox}}\clipbox{{\LLX} {\LLY} {\URX} {\URY}}{\color{blue}{\text}}
\begin{tikzpicture}[font=\sffamily,>=latex]
% Extensions lines for dimensions (drawn here to be below other material)
   \draw [gray,thin]
        (0,0)       -- +(-3.5ex,0)
        (0,\HEIGHT) -- +(-3.5ex,0)
        (0,-\DEPTH) -- +(-3.5ex,0)
        (\WIDTH,\HEIGHT) -- +(3.5ex,0)
        (\WIDTH,-\DEPTH) -- +(3.5ex,0)
        (0,-\DEPTH)      -- +(0,-3.5ex)
        (\WIDTH,-\DEPTH) -- +(0,-3.5ex)
   ;
% Text node:
   \node [inner sep=0pt,anchor=base west] {\usebox\mybox};
% Baseline
   \draw (0,0) -- (\WIDTH,0);% node [above,midway] {baseline};
%\draw [<-,shorten <=2pt] (0,0) -- (2ex,-2ex) -- +(.5ex,0) node [right=-0.5ex] {origin};
% Dimensions
   \draw [->] (-2.5ex,0) -- +(0,-\DEPTH)  node [midway,left] {depth};
   \draw [->] (-2.5ex,0) -- +(0, \HEIGHT) node [midway,left] {height};
   \draw [<->] (\WIDTH,-\DEPTH) ++(2.5ex,0) -- +(0,\DEPTH+\HEIGHT) node [midway,right] {totalheight};
   \draw [<->] (0,-\DEPTH) ++(0,-2.5ex) -- +(\WIDTH,0) node [midway,below] {width};
   \fill (-2.5ex,0) circle (.5pt);
   \begin{scope}[blue]
        \clip ([shift={(\LLX,\LLY)}]0,-\DEPTH) rectangle ([shift={(-\URX,-\URY)}]\WIDTH,\HEIGHT);
        \node [inner sep=0pt,anchor=base west,color=blue!50!white] {\text};
   \end{scope}
   \begin{scope}[blue]
        \draw ([shift={(\LLX,\LLY)}]0,-\DEPTH) rectangle ([shift={(-\URX,-\URY)}]\WIDTH,\HEIGHT);
        \draw (\LLX,0) -- ([shift={(-\URX,0)}]\WIDTH,0);
        \path [fill=white,draw] (\LLX,0) circle (1pt);
   \end{scope}
   \begin{scope}[blue!50!black] %,every node/.append style={inner sep=2pt}]
%\draw [->,.!45] (0,0) -- ++(-\URX+\WIDTH,-\URY+\HEIGHT);
%\draw [->,.!45] (0,0) -- ++(\LLX,\LLY-\DEPTH);
    \draw [->] (\LLX,0) -- (\LLX,\LLY-\DEPTH) node [right,midway] {\scriptsize lly};
    \draw [->] (0,-\DEPTH+\LLY) -- ++(\LLX,0) node [above,midway] {\scriptsize llx};
    \draw [->]
        ([shift={(-\URX,0)}]\WIDTH,0) --
        ([shift={(-\URX,-\URY)}]\WIDTH,\HEIGHT)
        node [right,midway] {\scriptsize ury}
    ;
    \draw [->]
        ([shift={(0,-\URY)}]0,\HEIGHT) --
        ([shift={(-\URX,-\URY)}]\WIDTH,\HEIGHT)
        node [above,midway] {\scriptsize urx}
    ;
   \end{scope}
% Box
   \draw [thick] (0,-\DEPTH) rectangle (\WIDTH,\HEIGHT);
% Origin
   \path [fill=white,draw=black] (0,0) circle (1pt);
% Center inner box
   \path let
        \p1 = (current bounding box.south west),
        \p2 = (current bounding box.north east),
        \p3 = (0,-\DEPTH),
        \p4 = (\WIDTH,\HEIGHT)
   in
        (\x3-\x2+\x4,\y3) rectangle (\x4+\x3-\x1,\y4);
\end{tikzpicture}
%<*standalone>
\end{document}
%</standalone>
